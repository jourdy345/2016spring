\title{Graduate School Pre-exam Solution}
\author{Daeyoung Lim}

\documentclass[answers]{exam}
\usepackage[left=3cm,right=3cm,top=3.5cm,bottom=2cm]{geometry}
\usepackage{amssymb,amsmath}
\usepackage[utf8]{inputenc}
\usepackage[T1]{fontenc}
\usepackage{lmodern}
\usepackage{enumerate}
\usepackage{listings}
\usepackage{courier}
\usepackage{cancel}
\usepackage{array}
\usepackage{courier}
\usepackage{booktabs}
\usepackage{titlesec}
\usepackage{enumitem}
\usepackage[usenames, dvipsnames]{color}
\setcounter{secnumdepth}{4}
\lstset{
         basicstyle=\footnotesize\ttfamily, % Standardschrift
         %numbers=left,               % Ort der Zeilennummern
         numberstyle=\tiny,          % Stil der Zeilennummern
         %stepnumber=2,               % Abstand zwischen den Zeilennummern
         numbersep=5pt,              % Abstand der Nummern zum Text
         tabsize=2,                  % Groesse von Tabs
         extendedchars=true,         %
         breaklines=true,            % Zeilen werden Umgebrochen
         keywordstyle=\color{red},
            frame=b,         
 %        keywordstyle=[1]\textbf,    % Stil der Keywords
 %        keywordstyle=[2]\textbf,    %
 %        keywordstyle=[3]\textbf,    %
 %        keywordstyle=[4]\textbf,   \sqrt{\sqrt{}} %
         stringstyle=\color{white}\ttfamily, % Farbe der String
         showspaces=false,           % Leerzeichen anzeigen ?
         showtabs=false,             % Tabs anzeigen ?
         xleftmargin=17pt,
         framexleftmargin=17pt,
         framexrightmargin=5pt,
         framexbottommargin=4pt,
         %backgroundcolor=\color{lightgray},
         showstringspaces=false      % Leerzeichen in Strings anzeigen ?        
 }
 \lstloadlanguages{% Check Dokumentation for further languages ...
         %[Visual]Basic
         %Pascal
         %C
         %C++
         %XML
         %HTML
         Java
 }
    %\DeclareCaptionFont{blue}{\color{blue}} 

\definecolor{myblue}{RGB}{72, 165, 226}
\definecolor{myorange}{RGB}{222, 141, 8}
\titleformat{\paragraph}
{\normalfont\normalsize\bfseries}{\theparagraph}{1em}{}
\titlespacing*{\paragraph}
{0pt}{3.25ex plus 1ex minus .2ex}{1.5ex plus .2ex}
\setlength{\heavyrulewidth}{1.5pt}
\setlength{\abovetopsep}{4pt}
\setlength{\parindent}{0mm}
\linespread{1.3}
\DeclareMathOperator{\sech}{sech}
\DeclareMathOperator{\csch}{csch}
\DeclareMathOperator{\argmin}{\arg\!\min}
\DeclareMathOperator{\Tr}{Tr}

\newcommand{\bs}{\boldsymbol}
\newcommand{\opn}{\operatorname}
%%%%%%%%%%%%%%%%%%%%%%%%%%%%%%%%%%%%%%%%%%%%%%%%%%%%%%%
% % We use newtheorem to define theorem-like structures
% %
% % Here are some common ones. . .
%%%%%%%%%%%%%%%%%%%%%%%%%%%%%%%%%%%%%%%%%%%%%%%%%%%%%%%
\newtheorem{theorem}{Theorem}
\newtheorem{lemma}{Lemma}
\newtheorem{proposition}{Proposition}
\newtheorem{scolium}{Scolium}   %% And a not so common one.
\newtheorem{definition}{Definition}
\newenvironment{proof}{{\sc Proof:}}{~\hfill QED}
\newenvironment{AMS}{}{}
\newenvironment{keywords}{}{}
%%%%%%%%%%%%%%%%%%%%%%%%%%%%%%%%%%%%%%%%%%%%%%%%%%%%%%%
% %   The first thanks indicates your affiliation
% %
% %  Just the name here.
% %
% % Your mailing address goes at the end.
% %
% % \thanks is also how you indicate grant support
% %
%%%%%%%%%%%%%%%%%%%%%%%%%%%%%%%%%%%%%%%%%%%%%%%%%%%%%%%


\begin{document}
\newpage
\firstpageheader{}{}{\bf\large Daeyoung Lim \\ Real Analysis \\ HW2}
\runningheader{Daeyoung Lim}{Real Analysis}{HW2}
\begin{questions}
   \question
   Prove M6. (You must use the assumption $\lambda\left(A_{1}\right) <\infty$.) If the $A_{k}$'s are measurable and $A_{1} \supset A_{2} \supset A_{3} \supset \cdots$, and if $\lambda\left(A_{1}\right) < \infty$, then
   $$
   \lambda\left(\bigcap\limits_{k=1}^{\infty}A_{k} \right) = \lim_{k\to\infty} \lambda\left(A_{k}\right).
   $$
   \begin{solution}
   In order to use M5, let us define $B_{k} = A_{1} \setminus A_{k}$. According to the Corollary, if $A_{1}, A_{k} \in \mathcal{L}_{0}$, then $A_{1}\setminus A_{k} \in \mathcal{L}_{0}$ as well. Also $B_{k}$ has the following property:
   $$
   B_{1} \subset B_{2} \subset B_{3} \subset \cdots.
   $$
   This is because
   \begin{align*}
   B_{1} &= A_{1} \setminus A_{1} = \emptyset\\
   B_{2} &= A_{1} \setminus A_{2} = A_{1} \cap A_{2}^{\mathsf{c}}\\
   &\vdots\\
   B_{k} &= A_{1} \setminus A_{k} = A_{1} \cap A_{k}^{\mathsf{c}}
   \end{align*}
   and $A_{1}^{\mathsf{c}} \subset A_{2}^{\mathsf{c}} \subset \cdots $.
   Therefore, we can apply M5 to $B_{k}$:
   $$
   \lambda\left(\bigcup_{k=1}^{\infty}B_{k}\right) = \lim_{k\to\infty} \lambda\left(B_{k}\right).
   $$
   Rewriting the LHS in terms of its original representation,
   \begin{align*}
   \bigcup_{k=1}^{\infty}B_{k} &= \left(A_{1}\cap A_{1}^{\mathsf{c}}\right) \cup \left(A_{1}\cap A_{1}^{\mathsf{c}}\right) \cup \cdots = A_{1} \cap \left(A_{1}^{\mathsf{c}}\cup A_{2}^{\mathsf{c}}\cup \cdots \right)\\
   &= A_{1} \cap \left(\bigcup_{k=1}^{\infty}A_{k}^{\mathsf{c}} \right)\\
   &= A_{1} \cap \left(\bigcap_{k=1}^{\infty}A_{k} \right)^{\mathsf{c}}\qquad \text{(By DeMorgan's Law)}\\
   &= A_{1}\setminus \left(\bigcap_{k=1}^{\infty}A_{k} \right)
   \end{align*}
   By M2, $\bigcap_{k=1}^{\infty}A_{k} \in \mathcal{L}$ and thus $A_{1}\setminus \bigcap_{k=1}^{\infty}A_{k} \in \mathcal{L}$. Now we will use M11 since $\bigcap_{k=1}^{\infty}A_{k} \subset A_{1}$:
   $$
   \lambda^{*}\left(\bigcap_{k=1}^{\infty}A_{k}\right) + \lambda_{*}\left(A_{1}\setminus \bigcap_{k=1}^{\infty}A_{k}\right) = \lambda\left(A_{1}\right).
   $$
   Because they are all measurable, outer measure, inner measure, and measure are all same:
   \begin{equation}
   \lambda\left(\bigcap_{k=1}^{\infty}A_{k}\right) + \lim_{k\to\infty}\lambda\left(B_{k}\right) = \lambda\left(A_{1}\right).
   \end{equation}
   Apply M11 again to $B_{k}=A_{1}\setminus A_{k}$ using the fact that $A_{k} \subset A_{1}$,
   $$
   \lambda^{*}\left(A_{k}\right) + \lambda_{*}\left(A_{1}\setminus A_{k}\right) = \lambda\left(A_{1}\right).
   $$
   Likewise, they are all measurable and thus inner measure, outer measure are all same. Taking the limit on both sides after rearrangement,
   \begin{equation}
      \lim_{k\to\infty}\lambda\left(A_{k}\right)  = \lambda\left(A_{1}\right) - \lim_{k\to\infty} \lambda\left(B_{k}\right).
   \end{equation}
   Plugging eqn (2) back to eqn (1), we obtain
   $$
      \lambda\left(\bigcap_{k=1}^{\infty}A_{k}\right) = \lim_{k\to\infty}\lambda\left(A_{k}\right).
   $$
   \end{solution}
   \question
   Let $A \subset \mathbb{R}^{n}$ be arbitrary. Prove that
   $$
   \lambda^{*}\left(A\right) = \inf \left\{\sum_{k=1}^{\infty}\lambda\left(I_{k}\right): A \subset \bigcup_{k=1}^{\infty}I_{k} \right\}
   $$
   where the $I_{k}$'s are special rectangles.\\
   (HINT: If $A \subset \bigcup_{k=1}^{\infty} I_{k}$, the $\lambda^{*}\left(A\right)\leq \sum_{k=1}^{\infty}\lambda\left(I_{k}\right)$ (why?). This will establish that $\lambda^{*}\left(A\right) \leq \inf \left\{\; \right\}$. To establish the reverse inequality, consider a well chosen $G \supset A$, and apply Problem 9.)
   \begin{solution}
   Let $I_{k}$ be nonoverlapping special rectangles. Then by *3,
   $$
   \lambda^{*}\left(A\right) \leq \lambda^{*}\left(\bigcup_{k=1}^{\infty}I_{k}\right).
   $$
   Since the measure of $\bigcup_{k=1}^{\infty}I_{k}$ is already defined in stage 1, the outer measure has to be the measure itself:
   $$
   \lambda^{*}\left(\bigcup_{k=1}^{\infty}I_{k}\right) = \lambda\left(\bigcup_{k=1}^{\infty}I_{k}\right)
   $$
   which is also $\sum_{k=1}^{\infty}\lambda\left(I_{k}\right)$. Therefore, we have obtained the first inequality
   $$
   \lambda^{*}\left(A\right) \leq \sum_{k=1}^{\infty}\lambda\left(I_{k}\right).
   $$
   Next, according to Problem 9, any nonempty open set $G \subset \mathbb{R}^{n}$ can be expressed as
   $$
   G = \bigcup_{k=1}^{\infty}I_{k},
   $$
   then, $A \subset G$. 
   \end{solution}
   \question
   Suppose that $A \cup B$ is measurable and that
   $$
   \lambda\left(A\cup B\right) = \lambda^{*}\left(A\right) + \lambda^{*}\left(B\right) < \infty.
   $$
   Prove that $A$ and $B$ are measurable.
   \begin{solution}
   By M11,
   \begin{align*}
   \lambda^{*}\left(B\right) + \lambda_{*}\left(\left(A\cup B\right)\setminus B\right) &= \lambda\left(A\cup B\right)\\
   &= \lambda^{*}\left(A\right) + \lambda^{*}\left(B\right).
   \end{align*}
   Note that $\left(A\cup B\right)\setminus B = A\setminus B$ using elementary set operations:
   \begin{align*}
      \left(A\cup B\right)\setminus B &= \left(A \cup B\right) \cap B^{\mathsf{c}}\\
      &= \left(A \cap B^{\mathsf{c}}\right) \cup \left(B \cap B^{\mathsf{c}}\right)\\
      &= \left(A \cap B^{\mathsf{c}}\right)\\
      &= A \setminus B.
   \end{align*}
   Therefore, $\lambda_{*}\left(A\setminus B\right) = \lambda^{*}\left(A\right)$ which by definition is
   $$
      \sup \left\{\lambda\left(K\right): K\subset A\setminus B, \text{ $K$ compact} \right\} = \inf\left\{\lambda\left(G\right): A \subset G, \text{ $G$ open} \right\}.
   $$
   For every compact set $K\subset A\setminus B$, it is obvious that $K \subset A$. Therefore, $K \subset A \subset G$, which lends itself directly to the theorem on approximation. Therefore, $A$ is measurable. Now using M11 once more as
   \begin{align*}
   \lambda^{*}\left(A\right) + \lambda_{*}\left(\left(A \cup B\right)\setminus A\right) &= \lambda\left(A \cup B\right)\\
   &= \lambda^{*}\left(A\right) + \lambda^{*}\left(B\right),
   \end{align*}
   we can arrive at the same conclusion for $B$.
   \end{solution}
   \question
   Prove that if $A$ and $B$ are measurable , then
   $$
   \lambda\left(A\right) + \lambda\left(B\right) = \lambda\left(A\cup B\right) + \lambda\left(A\cap B\right).
   $$
   \begin{solution}
   By M3, $A\setminus B, B\setminus A \in \mathcal{L}$ and by corollary, $A\cap B \in \mathcal{L}$. According to the theorem of countable additivity,
   \begin{align*}
   \lambda\left(A\right) &= \lambda\left(A\setminus B\right) + \lambda\left(A \cap B\right)\\
   \lambda\left(B\right) &= \lambda\left(B\setminus A\right) + \lambda\left(A \cap B\right)
   \end{align*}
   holds for disjoint sets $A\setminus B, A\cap B$ and $B\setminus A, A\cap B$. Therefore,
   $$
   \lambda\left(A\right) + \lambda\left(B\right) = \lambda\left(A\setminus B\right) + \lambda\left(B\setminus A\right) + 2\lambda\left(A\cap B\right).
   $$
   Because $A \setminus B, A \cap B, B\setminus A$ are all disjoint, the following holds:
   $$
   \lambda\left(\left(A\setminus B\right) \cup \left(B\setminus A\right) \cup \left(A\cap B\right) \right) = \lambda\left(A\setminus B\right) + \lambda\left(B\setminus A\right) + \lambda\left(A\cap B\right).
   $$
   Note that $\left(A\setminus B\right) \cup \left(B\setminus A\right) \cup \left(A\cap B\right) = A \cup B$. Therefore,
   $$
   \lambda\left(A\right) + \lambda\left(B\right) = \lambda\left(A \cup B\right) + \lambda\left(A\cap B\right).
   $$
   \end{solution}
   \question
   Prove that in general
   $$
   \lambda^{*}\left(A\right) + \lambda^{*}\left(B\right) \geq \lambda^{*}\left(A\cup B\right)+\lambda^{*}\left(A \cap B\right)
   $$
   and
   $$
   \lambda_{*}\left(A\right) + \lambda_{*}\left(B\right) \leq \lambda_{*}\left(A\cup B\right)+\lambda_{*}\left(A \cap B\right).
   $$
   \begin{solution}
   Recall the definition of an outer measure:
   $$
   \lambda^{*}\left(A\right) = \inf \left\{\lambda\left(G\right): A \subset G, \text{ $G$ open} \right\}.
   $$
   Then since $A \subset G$ and $G$ is open, *5 states that $G$ is measurable. Therefore, we can take 2 open sets $G_{1}, G_{2}$ for which
   \begin{align*}
   &A \subset G_{1}\\
   &B \subset G_{2}.
   \end{align*}
   Then, it is obvious that $A \cup B \subset G_{1}\cup G_{2}$ and $A \cap B \subset G_{1}\cap G_{2}$. Now we will use the result from the previous problem for measurable sets that
   $$
   \lambda\left(G_{1}\right) + \lambda\left(G_{2}\right) = \lambda\left(G_{1}\cup G_{2}\right) + \lambda\left(G_{1}\cap G_{2}\right).
   $$
   \end{solution}
   \question
   Prove that if $A$ is countable, then $\lambda\left(A\right)= 0$. (In particular, $\lambda\left(\mathbb{Q}\right)=0$.)
   \begin{solution}
   Let $A \subset \mathbb{R}^{n}$. Since a countable set can be indexed as a sequence, let $\left\{x_{k} \right\}_{k=1}^{\infty}$ be the indexed sequence of the set $A$. Then for every $x_{k}$, there exists a neighbourhood $B\left(x_{k}, 2^{-k}\epsilon\right)$ such that $B\left(x_{k},2^{-k}\epsilon\right) \subset I_{k}$ where
   $$
   I_{k} = \left[x_{k1}-2^{-k}\epsilon, x_{k1}+2^{-k}\epsilon\right]\times \left[x_{k2}-2^{-k}\epsilon, x_{k2}+2^{-k}\epsilon\right] \times \cdots \times \left[x_{kn}-2^{-k}\epsilon, x_{kn}+2^{-k}\epsilon\right].
   $$
   It holds that $A \subset \bigcup_{k=1}^{\infty}B\left(x_{k},2^{-k}\epsilon\right)$ and $\bigcup_{k=1}^{\infty}B\left(x_{k},2^{-k}\epsilon\right) \subset \bigcup_{k=1}^{\infty}I_{k}$. Therefore,
   $$
   \lambda\left(A\right) \leq \lambda\left(\bigcup_{k=1}^{\infty}B\left(x_{k},2^{-k}\epsilon\right)\right) \leq \lambda\left(\bigcup_{k=1}^{\infty}I_{k}\right)\leq \sum_{k=1}^{\infty}\lambda\left(I_{k}\right) 
   $$
   where
   \begin{align*}
   \sum_{k=1}^{\infty}\lambda\left(I_{k}\right) &= 2^{n}\epsilon\sum_{k=1}^{\infty}2^{-nk}\\
   &= 2^{n}\epsilon \dfrac{2^{-n}}{1-2^{-n}}\\
   &= \epsilon \frac{2^{n}}{2^{n}-1} < \left(2\epsilon\right)^{n}.
   \end{align*}
   Since $\epsilon$ is arbitrary, 
   $$
   \lambda\left(A\right) \leq \lambda\left(\bigcup_{k=1}^{\infty}B\left(x_{k},2^{-k}\epsilon\right)\right) \leq \lambda\left(\bigcup_{k=1}^{\infty}I_{k}\right)\leq \sum_{k=1}^{\infty}\lambda\left(I_{k}\right) \leq \left(2\epsilon\right)^{n} \leq 0.
   $$
   Therefore, $\lambda\left(A\right) = 0$.
   \end{solution}
   \question
   Let $a \in \mathbb{R}$ be fixed. Prove that
   $$
   \lambda\left(\left\{a \right\} \times \mathbb{R}^{n-1} \right) = 0.
   $$
   \begin{solution}
   We will use problem 9 and problem 27 to solve this. According problem 9, $\mathbb{R}^{n}$ can be covered with special rectangles having  a side length of 1, putting each on the lattice of integer length cubes. Since the set of integers is countable, we can recast it into a sequence $\left\{x_{s} \right\}_{s=1}^{\infty}$. 
   \end{solution}
   \question
   Prove that if $E\subset \mathbb{R}^{n}$ and $\lambda^{*}\left(E\right)<\infty $, then a measurable set $A$ exists such that
   $$
   E \subset A \text{ and } \lambda^{*}\left(E\right) = \lambda\left(A\right).
   $$
   (The set $A$ is called a \emph{measurable hull} of E.)\\
   (HINT: Choose open sets $G_{k}$ such that $\lambda\left(G_{k}\right) < \lambda^{*}\left(E\right)+k^{-1}$ and $E \subset G_{k}$. Then let $A = \bigcap_{k=1}^{\infty}G_{k}$.)
   \begin{solution}
   Using the hint, since $G_{k}$ are open sets, we can use property *5 and say
   $$
   \lambda\left(G_{k}\right) = \lambda^{*}\left(G_{k}\right) = \lambda_{*}\left(G_{k}\right).
   $$
   Because we chose $G_{k}$ such that $\lambda\left(G_{K}\right) < \lambda^{*}\left(E\right) + k^{-1}$, it is obvious that $\lambda\left(G\right) < \infty$. By definition, $G_{k} \in \mathcal{L}_{0}$ and thus $G_{k} \in \mathcal{L}$. Now we rely on the fact that $\mathcal{L}$ is a $\sigma$-algebra. Then,
   $$
   A = \bigcap_{k=1}^{\infty}G_{k} \in \mathcal{L}.
   $$
   Since $E \subset G_{k}$ for all $k$, it is obvious that $E \subset \bigcap_{k=1}^{\infty}G_{k}$. Therefore,
   $$
   \lambda\left(\bigcap_{k=1}^{\infty}G_{k}\right) \geq \lambda^{*}\left(E\right).
   $$
   For the other direction, we should note that $\lambda\left(\bigcap_{k=1}^{\infty}G_{k}\right) \leq \lambda\left(G_{k}\right)$ since $\bigcap_{k=1}^{\infty}G_{k} \subset G_{k}$ and *2. We have chosen $G_{k}$ such that $\lambda\left(G_{k}\right) < \lambda^{*}\left(E\right) +k^{-1}$, the following holds:
   $$
   \lambda\left(\bigcap_{k=1}^{\infty}G_{k}\right) \leq \lambda\left(G_{k}\right) < \lambda^{*}\left(E\right) + k^{-1}.
   $$
   Since $k^{-1} \to 0$ as $k \to \infty$, we can conclude that
   $$
   \lambda\left(\bigcap_{k=1}^{\infty}G_{k}\right) \leq \lambda^{*}\left(E\right).
   $$
   Combining both inequalities,
   $$
   \lambda\left(A\right) = \lambda^{*}\left(E\right).
   $$
   \end{solution}
   \question
   Let $\lambda^{*}\left(E\right) < \infty$, $E \subset A \in \mathcal{L}$. Prove that $A$ is a measurable hull of $E \iff \lambda_{*}\left(A\setminus E\right) = 0$.
   \begin{solution}
   $\left(\Rightarrow\right)$\\
   If $A$ is a measurable hull of $E$, by M11 and the result of the previous problem,
   \begin{align*}
   \lambda^{*}\left(E\right) + \lambda_{*}\left(A\setminus E\right) &= \lambda\left(A\right)\\
   \lambda\left(A\right) &= \lambda^{*}\left(E\right).
   \end{align*}
   Combining two equations, we obtain that
   $$
   \lambda_{*}\left(A\setminus E\right) = 0.
   $$
   $\left(\Leftarrow\right)$\\
   If $\lambda_{*}\left(A\setminus E\right) = 0$, by M11
   \begin{align*}
   \lambda^{*}\left(E\right) = \lambda\left(A\right).
   \end{align*}
   By the result from the previous problem, we call this set $A$, the measurable hull of $E$.
   \end{solution}
   \question
   Prove that if $E_{1} \subset E_{2} \subset E_{3} \subset \cdots $, then
   $$
   \lambda^{*}\left(\bigcup_{k=1}^{\infty}E_{k} \right) = \lim_{k\to\infty}\lambda^{*}\left(E_{k}\right).
   $$
   (HINT: Let $A_{k}$ be a measurable hull of $E_{k}$, and define $B_{k}=\bigcap_{j=k}^{\infty}A_{j}$. Then $B_{k}$ is also a measurable hull of $E_{k}$, and furthermore $B_{1} \subset B_{2} \subset B_{3} \subset \cdots$. Apply Property M5.)
   \begin{solution}

   \end{solution}
   \question
   Suppose $A_{1}, A_{2}, A_{3}, \ldots$ are measurable sets and
   $$
   \sum_{k=1}^{\infty}\lambda\left(A_{k}\right) <\infty.
   $$
   Prove that $\lambda\left(\limsup A_{k}\right) = 0$.
   \begin{solution}

   \end{solution}
\end{questions}
\end{document}
