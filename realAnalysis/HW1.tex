\title{Graduate School Pre-exam Solution}
\author{Daeyoung Lim}

\documentclass[answers]{exam}
\usepackage[left=3cm,right=3cm,top=3.5cm,bottom=2cm]{geometry}
\usepackage{amssymb,amsmath}
\usepackage[utf8]{inputenc}
\usepackage[T1]{fontenc}
\usepackage{lmodern}
\usepackage{enumerate}
\usepackage{listings}
\usepackage{courier}
\usepackage{cancel}
\usepackage{array}
\usepackage{courier}
\usepackage{booktabs}
\usepackage{titlesec}
\usepackage{enumitem}
\usepackage[usenames, dvipsnames]{color}
\setcounter{secnumdepth}{4}
\lstset{
         basicstyle=\footnotesize\ttfamily, % Standardschrift
         %numbers=left,               % Ort der Zeilennummern
         numberstyle=\tiny,          % Stil der Zeilennummern
         %stepnumber=2,               % Abstand zwischen den Zeilennummern
         numbersep=5pt,              % Abstand der Nummern zum Text
         tabsize=2,                  % Groesse von Tabs
         extendedchars=true,         %
         breaklines=true,            % Zeilen werden Umgebrochen
         keywordstyle=\color{red},
            frame=b,         
 %        keywordstyle=[1]\textbf,    % Stil der Keywords
 %        keywordstyle=[2]\textbf,    %
 %        keywordstyle=[3]\textbf,    %
 %        keywordstyle=[4]\textbf,   \sqrt{\sqrt{}} %
         stringstyle=\color{white}\ttfamily, % Farbe der String
         showspaces=false,           % Leerzeichen anzeigen ?
         showtabs=false,             % Tabs anzeigen ?
         xleftmargin=17pt,
         framexleftmargin=17pt,
         framexrightmargin=5pt,
         framexbottommargin=4pt,
         %backgroundcolor=\color{lightgray},
         showstringspaces=false      % Leerzeichen in Strings anzeigen ?        
 }
 \lstloadlanguages{% Check Dokumentation for further languages ...
         %[Visual]Basic
         %Pascal
         %C
         %C++
         %XML
         %HTML
         Java
 }
    %\DeclareCaptionFont{blue}{\color{blue}} 

\definecolor{myblue}{RGB}{72, 165, 226}
\definecolor{myorange}{RGB}{222, 141, 8}
\titleformat{\paragraph}
{\normalfont\normalsize\bfseries}{\theparagraph}{1em}{}
\titlespacing*{\paragraph}
{0pt}{3.25ex plus 1ex minus .2ex}{1.5ex plus .2ex}
\setlength{\heavyrulewidth}{1.5pt}
\setlength{\abovetopsep}{4pt}
\setlength{\parindent}{0mm}
\linespread{1.3}
\DeclareMathOperator{\sech}{sech}
\DeclareMathOperator{\csch}{csch}
\DeclareMathOperator{\argmin}{\arg\!\min}
\DeclareMathOperator{\Tr}{Tr}

\newcommand{\bs}{\boldsymbol}
\newcommand{\opn}{\operatorname}
%%%%%%%%%%%%%%%%%%%%%%%%%%%%%%%%%%%%%%%%%%%%%%%%%%%%%%%
% % We use newtheorem to define theorem-like structures
% %
% % Here are some common ones. . .
%%%%%%%%%%%%%%%%%%%%%%%%%%%%%%%%%%%%%%%%%%%%%%%%%%%%%%%
\newtheorem{theorem}{Theorem}
\newtheorem{lemma}{Lemma}
\newtheorem{proposition}{Proposition}
\newtheorem{scolium}{Scolium}   %% And a not so common one.
\newtheorem{definition}{Definition}
\newenvironment{proof}{{\sc Proof:}}{~\hfill QED}
\newenvironment{AMS}{}{}
\newenvironment{keywords}{}{}
%%%%%%%%%%%%%%%%%%%%%%%%%%%%%%%%%%%%%%%%%%%%%%%%%%%%%%%
% %   The first thanks indicates your affiliation
% %
% %  Just the name here.
% %
% % Your mailing address goes at the end.
% %
% % \thanks is also how you indicate grant support
% %
%%%%%%%%%%%%%%%%%%%%%%%%%%%%%%%%%%%%%%%%%%%%%%%%%%%%%%%


\begin{document}
\newpage
\firstpageheader{}{}{\bf\large Daeyoung Lim \\ Real Analysis \\ HW1}
\runningheader{Daeyoung Lim}{Real Analysis}{HW1}
\begin{questions}
   \question
   If $G$ is open and $P$ is a special polygon with $P \subset G$, prove there exists a special polygon $P'$ such that $P \subset P' \subset G$ and $\lambda\left(P\right) <\lambda\left(P'\right)$.
   \begin{solution}
      Recall the definition of a measure on an open set $G$:
      $$
         \lambda\left(G\right) := \sup \left\{\lambda\left(P\right): P \subset G, P \text{ is a special polygon} \right\}.
      $$
      We have two occasions: 
      \begin{enumerate}
      \item $\lambda\left(G\right) < \infty$.
      \item $\lambda\left(G\right) = \infty$. \end{enumerate}
      Let's address the first case.
      \begin{enumerate}
         \item If $\lambda\left(G\right) < \infty$, according to the definition of \emph{supremum}, it satisfies the following property:
         $$
            \lambda\left(G\right) - \epsilon = \lambda\left(P\right)
         $$
         for some $\epsilon > 0$. Then by the definition of \emph{supremum}, there exists an element $\lambda\left(P'\right) \in \left\{\lambda\left(P\right): P \subset G, P \text{ is a special polygon} \right\}$ such that
         $$
            \underbrace{\lambda\left(G\right) - \epsilon < \lambda\left(P'\right)}_{\lambda\left(P\right) < \lambda\left(P'\right)} < \lambda\left(G\right).
         $$
         \item If $\lambda\left(G\right) = \infty$, the set $\left\{\lambda\left(P\right): P \subset G, P\text{ is a special polygon} \right\}$ is not bounded above. Therefore, it immediately follows that there exists an element $$\lambda\left(P'\right) \in \left\{\lambda\left(P\right): P \subset G, P\text{ is a special polygon} \right\}$$ such that $\lambda\left(P\right) < \lambda\left(P'\right)$.
      \end{enumerate}
   \end{solution}
   \question
   \textbf{(a)} Prove that if $G$ is a bounded open set, then $\lambda\left(G\right)<\infty$.\\
   \textbf{(b)} In the plane $\mathbb{R}^{2}$ let
   $$
      G = \left\{\left(x,y\right): 1<x \text{ and } 0<y<\frac{1}{x} \right\}.
   $$
   Prove that $\lambda\left(G\right) = \infty$\\
   \textbf{(c)} In the plane $\mathbb{R}^{2}$ let
   $$
      G = \left\{\left(x,y\right): 0<x \text{ and } 0<y<e^{-x} \right\}.
   $$
   Prove that $\lambda\left(G\right)=1$. \\
   \textbf{(d)} In the plane $\mathbb{R}^{2}$ let
   $$
      G = \left\{\left(x,y\right): 1<x \text{ and } 0<y<x^{-a} \right\},
   $$
   where $a$ is a real number satisfying $a>1$. Prove that $\lambda\left(G\right) = 1/\left(a-1\right)$.
   \begin{solution}
      \textbf{(a)} Let $G'$ be the set of all limit points of $G$. Then the closure $\overline{G} = G \bigcup G'$ is compact because it is a closed and bounded set. Since $\overline{G}$ is compact, there exists a finite subcover $\bigcup_{i=1}^{k}V_{i}$ of $\overline{G}$. If a set is closed and bounded in $\mathbb{R}^{n}$, then it can be contained in $I$ for some $n$-cell $I$, i.e., $\overline{G} \subset I$. Because $n$-cell is a special rectangle, $\lambda\left(I\right)$ is well-defined to be finite. Thus,
      $$
         \lambda\left(G\right) \leq \lambda\left(\overline{G}\right) \leq \lambda\left(I\right) < \infty.
      $$
   \end{solution}

   \begin{solution}

   \end{solution}

   \begin{solution}

   \end{solution}

   \begin{solution}

   \end{solution}
   \question
   Let $G_{i}, i \in \mathcal{I}$, be a collection of disjoint open sets in $\mathbb{R}^{n}$. Prove that only countably many of these sets are nonempty.
   \begin{solution}
      We can use the fact that $\mathbb{Q}^{n}$ is dense in $\mathbb{R}^{n}$, which is put another way as such: every neighbourhood in $\mathbb{R}^{n}$ contains a point of $\mathbb{Q}^{n}$. Therefore, if $G_{i}$ is a nonempty open set, every point in $G_{i}$ is an interior point with a neighbourhood with a point in $\mathbb{Q}^{n}$ contained in $G_{i}$. Since $G_{i}$ are disjoint, they must not share a point in $\mathbb{Q}^{n}$, which concludes that the number of nonempty open sets in the collection $\left\{G_{i}\right\}$ is as large as the cardinality of $\mathbb{Q}^{n}$. $\mathbb{Q}^{n}$ has the same cardinality as $\mathbb{N}^{n}$ thereby lending itself to the definition of \emph{countability}. 
   \end{solution}
   \question
   \emph{The structure of open sets in} $\mathbb{R}$.\\
   Prove that every nonempty open subset of $\mathbb{R}$ can be expressed as a countable disjoint union of open intervals:
   $$
      G = \bigcup\limits_{k} \left(a_{k}, b_{k}\right),
   $$
   where the range on $k$ can be finite or infinite. Furthermore, show that this expression is unique except for the numbering of the component intervals.
   \begin{solution}
      For every point $x \in G$, we define $A_{x}$ to be the largest interval contained within $G$. We can construct $A_{x}$ by making use of supremum and infimum.
      \begin{align*}
         b_{x} &:= \sup\left\{b: b>x \text{ and } \left(x,b\right) \subset G \right\}\\
         a_{x} &:= \inf \left\{a: a < x \text{ and } \left(a, x\right) \subset G \right\}
      \end{align*}
      Then $A_{x} := \left(a_{x}, b_{x}\right)$. By construction, $G = \bigcup_{x\in G} A_{x}$. Now we are left with two problems: (1) $A_{x}$ is uniquely defined. (2) The collection $\left\{A_{x} \right\}_{x\in G}$ is countable.
      \begin{enumerate}
         \item Suppose there exist two distinct points $x,x' \in G$ whose defined $A_{x}$ and $A_{x'}$ overlap, i.e., $A_{x} \bigcap A_{x'} \neq \emptyset$. It follows that $A_{x} \bigcup A_{x'} \subset A_{x}$ since $A_{x}$ is by definition the largest interval containing $x$ within $G$. Likewise, $A_{x} \bigcup A_{x'} \subset A_{x'}$. Together, they suggest $A_{x} = A_{x'}$. This implies that if two $A_{x}, A_{x'}$ overlap, they do not overlap partially. They overlap entirely which makes them the same interval. Otherwise, they are disjoint, which makes each of them unique in the collection $\left\{A_{x} \right\}_{x\in G}$.
         \item Recall that the set of rational numbers $\mathbb{Q}$ is dense in $\mathbb{R}$. This states that every set in the collection $\left\{A_{x} \right\}_{x\in G}$ has at least one rational number. Since they are disjoint, there exists an injection $\left\{A_{x} \right\}_{x\in G} \mapsto \mathbb{Q}$. Therefore, the cardinality of the collection $\left\{A_{x} \right\}_{x\in G}$ is at its largest the same as that of $\mathbb{Q}$. 
      \end{enumerate}
   \end{solution}
   \question
   In the notation of the previous problem, prove that $\lambda\left(G\right) = \sum_{k} \left(b_{k}-a_{k}\right)$.
   \begin{solution}
      By property O6, for disjoint sets $A_{x}$,
      $$
         \lambda\left(G\right) = \lambda\left(\bigcup_{x\in G}A_{x}\right) = \sum_{x\in G}\lambda\left(A_{x}\right).
      $$
      Since $\lambda\left(A_{x}\right)$ is defined to be $b_{x} - a_{x}$, it follows that
      $$
         \lambda\left(G\right) = \sum_{x}\left(b_{x}-a_{x}\right).
      $$
   \end{solution}
   \question
   Prove that the open disk $B\left(0,1\right)$ in $\mathbb{R}^{2}$ cannot be expressed as a disjoint union of open rectangles.
   \begin{solution}
      Recall the fact that a unit ball is a \emph{connected} set, i.e., it is not a union of two nonempty separated sets. However, if we suppose nonempty disjoint open rectangles $\left\{G_{k} \right\}_{k=1}^{\infty}$ can be coupled to create the unit ball $B\left(0,1\right)$, then it violates the connectedness of a unit ball. Let $A_{1} = G_{1}$ and $A_{2} = \bigcup_{i=2}^{\infty} G_{i}$. Then $A_{1} \bigcap A_{2} = \emptyset$ but $A_{1} \bigcup A_{2} = B\left(0,1\right)$, which is a contradiction.
   \end{solution}

   \question
   Prove that every nonempty open subset of $\mathbb{R}^{n}$ can be expressed as a countable union of nonoverlapping special rectangles, which may be taken to be cubes:
   $$
      G = \bigcup_{k=1}^{\infty}I_{k}.
   $$
   The range on $k$ must be infinite. Why?\\
   (HINT: First pave $\mathbb{R}^{n}$ with cubes of side 1. Select those cubes which are contained in $G$. Then bisect the sides of the remaining cubes to obtain cubes with side $1/2$. Select those cubes which are contained in $G$.)
   \begin{solution}

   \end{solution}
   
   \question
   Let $\epsilon > 0$. Prove that there exists an open set $G \subset \mathbb{R}$ such that $\mathbb{Q} \subset G$ and $\lambda\left(G\right) < \epsilon$. (This result will probably surprise you: Although $G$ is open and contains every rational number, ``most'' of $\mathbb{R}$ is in $G^{\mathsf{c}}$.)
   \begin{solution}
      From the fact that $\mathbb{Q}$ is countable, we can construct a sequence $\left\{x_{n} \right\}, n \in \mathbb{Z}^{+}$ for which a bijection $f$ exists with $\mathbb{Q}$. Therefore, if we take sequence to be equivalent to the set of rational numbers, take a neighbourhood around each element in $\left\{x_{n} \right\}$ and denote it by $B\left(x_{n}, \epsilon/3^{n}\right)$ for some $\epsilon>0$. In $\mathbb{R}$, a neighbourhood is an open interval. Once we set $G$ to be the union of these neighbourhoods, it is then followed by
      $$
         \mathbb{Q} \subset \bigcup_{n=1}^{\infty}B\left(x_{n}, \frac{\epsilon}{3^{n}}\right).
      $$
      An infinite union of open sets is also an open set. Therefore, $G$ is also open. Since
      $$
         \lambda\left(G\right) = \lambda\left(\bigcup_{n=1}^{\infty} B\left(x_{n}, \frac{\epsilon}{3^{n}}\right) \right) \leq \sum_{n=1}^{\infty}\lambda\left(B\left(x_{n}, \frac{\epsilon}{3^{n}}\right)\right) = \sum_{n=1}^{\infty} \frac{\epsilon}{3^{n}},
      $$
      $\lambda\left(G\right) \leq \epsilon/2$.
   \end{solution}
   \question
   Use your method of working Problem 12 to give a proof that $\mathbb{R}$ is uncountable (cf. Section 1B).
   \begin{solution}
      Assume $\mathbb{R}$ is countable. Then using the method in the previous problem, we can construct a sequence with a bijection with $\mathbb{Z}^{+}$. Equally, there exists an open cover of countably many open intervals for $\mathbb{R}$. From this, there exists $\epsilon > 0$ which satisfies the following relation:
      $$
         \lambda\left(\mathbb{R}\right) < \epsilon.
      $$
      This is a violation of the property O3, stating $\lambda\left(\mathbb{R}\right)=\infty$.
   \end{solution}
\end{questions}
\end{document}
