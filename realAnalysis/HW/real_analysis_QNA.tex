\documentclass[11pt]{article}
\usepackage{amsmath,amssymb,amsthm}
\usepackage{times}
\usepackage{newtxmath}
\usepackage[utf8]{inputenc}
\addtolength{\evensidemargin}{-.5in}
\addtolength{\oddsidemargin}{-.5in}
\addtolength{\textwidth}{0.8in}
\addtolength{\textheight}{0.8in}
\addtolength{\topmargin}{-.4in}
\newtheoremstyle{quest}{\topsep}{\topsep}{}{}{\bfseries}{}{ }{\thmname{#1}\thmnote{ #3}.}
\theoremstyle{quest}
\newtheorem*{definition}{Definition}
\newtheorem*{theorem}{Theorem}
\newtheorem*{question}{Question}
\newtheorem*{exercise}{Exercise}
\newtheorem*{challengeproblem}{Challenge Problem}
\newcommand{\name}{%%%%%%%%%%%%%%%%%%
%%%%%%%%%%%%%%%%%%%%%%%%%%%%%%
%%%%%%%%%%%%%%%%%%%%%%%%%%%%%%
%% put your name here, so we know who to give credit to %%
Daeyoung Lim
}%%%%%%%%%%%%%%%%%%%%%%%%%%%%%%
% \newcommand{\hw}{%%%%%%%%%%%%%%%%%%%%
% %% and which homework assignment is it? %%%%%%%%%
% %% put the correct number below              %%%%%%%%%
% %%%%%%%%%%%%%%%%%%%%%%%%%%%%%%
% 1
% }
%%%%%%%%%%%%%%%%%%%%%%%%%%%%%%
%%%%%%%%%%%%%%%%%%%%%%%%%%%%%%
%%%%%%%%%%%%%%%%%%%%%%%%%%%%%%
\title{\vspace{-50pt}
\huge Real Analysis\hfill Solutions}
\author{Daeyoung Lim}
\date{}
\pagestyle{myheadings}
% \markright{\name\hfill Homework \hw\qquad\hfill}

%% If you want to define a new command, you can do it like this:
\newcommand{\Q}{\mathbb{Q}}
\newcommand{\R}{\mathbb{R}}
\newcommand{\Z}{\mathbb{Z}}
\newcommand{\C}{\mathbb{C}}

%% If you want to use a function like ''sin'' or ''cos'', you can do it like this
%% (we probably won't have much use for this)
% \DeclareMathOperator{\sin}{sin}   %% just an example (it's already defined)


\begin{document}
\maketitle
\fontfamily{ptm}
\section*{Chapter 10. $L^{p} Spaces$}
\begin{question}[10-1]
  If $1 < p < \infty$ and $a \geq 0$, $b \geq 0$, prove that
  \begin{equation}
    ab \leq \frac{a^{p}}{p} + \frac{b^{q}}{q},
  \end{equation}
  with equality if and only if $a^{p}=b^{q}$.
\end{question}

\begin{proof}
  Let $b$ be fixed and define a function
  \begin{equation}
    f\left(a\right) = ab-\frac{a^{p}}{p}
  \end{equation}
  and maximize the function using elementary calculus.
  \begin{align*}
    f'\left(a\right) &= b - a^{p-1}=0\\
    a &= b^{1/\left(p-1\right)}
  \end{align*}
  Therefore, the function is maximized when $a = b^{1/\left(p-1\right)}$. Plugging in,
  \begin{align*}
    f\left(b^{1/\left(p-1\right)}\right) &= b^{p/\left(p-1\right)} - \frac{b^{p/\left(1-p\right)}}{p}\\
    &= \frac{p-1}{p}b^{\frac{p}{p-1}}
  \end{align*}
  is the maximum value. Therefore, we can conclude that
  \begin{equation}
    ab-\frac{a^{p}}{p} \leq \frac{b^{q}}{q}
  \end{equation}
  by defining $q = p/\left(p-1\right)$.
\end{proof}

\begin{question}[10-2]
  Assume $1 < p_{k} < \infty$ for $k=1, \ldots , N$, and $\sum_{k=1}^{N}1/p_{k}=1$. Prove that
  \begin{equation}\label{prob10-2}
    \left| \int_{\mathcal{X}} f_{1}f_{2}\cdots f_{N}\,d\mu \right| \leq \left\|f_{1}\right\|_{p_{1}} \left\|f_{2}\right\|_{p_{2}}\cdots \left\|f_{N}\right\|_{p_{N}}.
  \end{equation}
\end{question}
\begin{proof}
  If $N=2$, the problem is reduced to the classical Hölder's inequality. Therefore, we will assume \ref{prob10-2} holds for some $N > 2$ and then show that it also holds for $N +1$. First, let us define
  $$
    q^{N+1} = \frac{p_{N+1}}{p_{N+1}-1}
  $$
  and for $i = 1, \dots , N$, define
  $$
    r_{i} = p_{i} \cdot \left(1-\frac{1}{p_{N+1}}\right).
  $$
  Then
  \begin{align*}
    \frac{1}{p_{N+1}} + \frac{1}{q_{N+1}} &= 1\\
    \sum_{i=1}^{N}\frac{1}{r_{i}} &= 1\\
    q_{N+1}\cdot r_{i} &= p_{i}.
  \end{align*}

  Applying Hölder's inequality to $f = \prod_{i=1}^{N}f_{i}$ and $g=f_{N+1}$, we find:
  \begin{align}
    \int_{\mathcal{X}}\left|\prod_{i=1}^{N+1}f_{i}\right|\,d\mu &\leq \left\|f_{N+1}\right\|_{p_{N+1}}\cdot \left\|\prod_{i=1}^{N}f_{i}\right\|_{q_{n}}\\
    &= \left\|f_{n}\right\|p_{n} \cdots \left(\int_{\mathcal{X}}\prod_{i=1}^{N}\left|f_{i}^{q_{N+1}}\right| \, d\mu\right)^{1/q_{N+1}}\\
    &\leq \left\|f_{N+1}\right\|_{p_{N+1}}\cdot \left(\prod_{i=1}^{N}\left\|f_{i}^{q_{N+1}}\right\|_{r_{i}}\right)^{1/q_{N+1}}\\
    =\prod_{i=1}^{N+1}\left\|f_{i}\right\|_{p_{i}}
  \end{align}
\end{proof}

\begin{question}[10-3]
  Assume in Hölder's inequality $f \geq 0, \, g \leq 0$, and 
  $$\label{prob10-3}
    \int_{\mathcal{X}}fg\,d\mu = \left\|f\right\|_{p}\left\|g\right\|_{q}.
  $$
  Prove that $f\left(x\right)^{p}=g\left(x\right)^{q}$ $\mu$-a.e., to within a multiplicative constant.
\end{question}
\begin{proof}
  If \ref{prob10-3} is true, then it also indicates
  $$
    \int_{\mathcal{X}}\bar{f}\bar{g}\,d\mu = 1
  $$
  where $\bar{f}=\frac{f}{\left\|f\right\|_{p}}$ and $\bar{g}=\frac{g}{\left\|g\right\|_{q}}$. It also translates to
  $$
    \int_{\mathcal{X}}\bar{f}\bar{g}\,d\mu = \frac{1}{p}\int_{\mathcal{X}}\bar{f}^{p}\,d\mu + \frac{1}{q}\int_{\mathcal{X}}\bar{g}^{q}\,d\mu.
  $$
  Therefore, $\bar{f}\bar{g}=\bar{f^{p}}/p + \bar{g^{q}}/q$ $\mu$-a.e. Since this is a case of \emph{Young's inequality}, the equality holds if and only if $\bar{f}^{p}=\bar{g}_{q}$ $\mu$-a.e. 
\end{proof}

\begin{question}[10-7]
  Suppose $1 \leq p < r < q < \infty$. Prove that $L^{p} \cap L^{q} \subset L^{r}$.
\end{question}
\begin{proof}
  Since $\frac{1}{q} < \frac{1}{r} < \frac{1}{p}$, there exists a unique $\theta$ such that
  $$
    \frac{1}{r} = \frac{\theta}{p} + \frac{1-\theta}{q}.
  $$
  The number $\theta$ satisfies $0 < \theta < 1$ and equals
  $$
    \theta = \dfrac{\dfrac{1}{r} - \dfrac{1}{q}}{\dfrac{1}{p}-\dfrac{1}{q}}, \qquad 1-\theta = \dfrac{\dfrac{1}{r}-\dfrac{1}{r}}{\dfrac{1}{p}-\dfrac{1}{q}}.
  $$
  It follows that $1 = \frac{r\theta}{p}+\frac{r\left(1-\theta\right)}{q}$ which indicates that $p/\left(r\theta\right)$ and $q/\left(r\left(1-\theta\right)\right)$ are conjugate exponents. Thus by Hölder's inequality,
  \begin{align}
    \left\|f\right\|_{r} &= \left\|f^{\theta}f^{1-\theta}\right\|_{r}\\
    &=\left\|f^{r\theta}f^{r\left(1-\theta\right)}\right\|_{1}^{1/r}\\
    &\leq \left(\left\|f^{r\theta}\right\|_{\frac{p}{r\theta}}\left\|f^{r\left(1-\theta\right)}\right\|_{\frac{q}{r\left(1-\theta\right)}}\right)^{1/r}\\
    &= \left(\left\|f\right\|_{p}^{r\theta}\left\|f\right\|_{q}^{r\left(1-\theta\right)}\right)^{1/r}\\
    &=\left\|f\right\|_{p}^{\theta}\left\|f\right\|_{q}^{1-\theta} < \infty.
  \end{align}
Therefore, $L^{r}$ norm is also finite for all $f \in L^{p}\cap L^{q}$.
\end{proof}

\section*{For the journal}

We have similar environments for Theorems, Exercises, and Challenge Problems.

\begin{theorem}[1.32]
  You should restate the theorem here.
\end{theorem}
\begin{proof}
  Insert your proof here.
\end{proof}


\begin{exercise}[1.22]

\end{exercise}
\begin{proof}

\end{proof}


\begin{challengeproblem}[1.14]

\end{challengeproblem}
\begin{proof}

\end{proof}




\section*{Writing in math mode}
When you want to write something in ``math mode'', you should enclose it in dollar signs: $x+y=a^2+b^2$.
If it's an important equation and you want to set it off, you can do it like this: \[x = \frac{-b \pm \sqrt{b^2-4ac} }{2a}\]
If you have a chain of deductions, you can line them up like this. (The ampersand \& tells it where to line up---usually you will want it right before the equals sign. You should have an ampersand in each line. The double-backslash $\backslash\backslash$ tells it to move to the next line.)
\begin{align*}
  2x &= 2y-10&\\
  x &=y-5 &\\
  z+x &= y-5+z
\end{align*}
If you want to include justifications, one way is like this:
\begin{align*}
  2x &= 2y-10 & (A5)\\
  x &=y-5  & (A4)\\
  z+x &= y-5+z & (W)
\end{align*}
You can also include words:

\begin{align*}
  2x &= 2y-10 & \text{by commutativity} \\
  x &=y-5  & \text{since $y$ is the GCD}\\
  z+x &= y-5+z & \text{because 2 is prime}
\end{align*}

Here are some symbols that might come up in the homeworks and journals (but remember you can view the source of the scripts and homeworks on the course webpage).

Let $x\in \Z[i]$ and $y\in \Q[i]$. Then $x^2+y^2\in \Q[i]$. If you have two numbers $n_1\in \Z$ and $n_2\in \Z$, then you should be satisfied. Many students don't have it so good. You can ask whether $n_1 < n_2$, or even if $n_1 \leq n_2$. I wonder if $a|b$? Greek letters: $\alpha$, $\beta$, $\gamma$, $\pi$, $\Gamma$, $\Delta$.

Maybe you want to talk about the set $S=\{ x\in \Z | x^2 = 2\}$, or maybe the set $T=\{ x\in \Z | x^2 \text{ is even}\}$. Did you notice that $S\subseteq T$? It's also true that $S \neq T$.

You might need to write fractions like $\frac{p}{q}$, or even $\frac{a}{a^2+b^2}$. If this fractions are too small, you can display them like this (see how I made those big parentheses?):
\[\left(\frac{p}{q}\right)\cdot \left(\frac{t}{u}\right) = \frac{a}{a^2+b^2}\]
If you can't figure out a symbol, you should google ``Detexify''. All you have to do is draw the symbol using your mouse, and it tells you the latex command.
%%%% don't delete the last line!
\end{document}