\documentclass[11pt]{article}

% This first part of the file is called the PREAMBLE. It includes
% customizations and command definitions. The preamble is everything
% between \documentclass and \begin{document}.

\usepackage[margin=0.6in]{geometry} % set the margins to 1in on all sides
\usepackage{graphicx} % to include figures
\usepackage{amsmath} % great math stuff
\usepackage{amsfonts} % for blackboard bold, etc
\usepackage{amsthm} % better theorem environments
% various theorems, numbered by section
\usepackage{amssymb}
\usepackage[utf8]{inputenc}
\usepackage{booktabs}
\usepackage{array}
\usepackage{courier}
\usepackage[usenames, dvipsnames]{color}
\usepackage{titlesec}
\usepackage{empheq}
\usepackage{tikz}
% \usepackage{mtpro2}
% \usepackage{pslatex}


\newcommand\encircle[1]{%
  \tikz[baseline=(X.base)] 
    \node (X) [draw, shape=circle, inner sep=0] {\strut #1};}
 
% Command "alignedbox{}{}" for a box within an align environment
% Source: http://www.latex-community.org/forum/viewtopic.php?f=46&t=8144
\newlength\dlf  % Define a new measure, dlf
\newcommand\alignedbox[2]{
% Argument #1 = before & if there were no box (lhs)
% Argument #2 = after & if there were no box (rhs)
&  % Alignment sign of the line
{
\settowidth\dlf{$\displaystyle #1$}  
    % The width of \dlf is the width of the lhs, with a displaystyle font
\addtolength\dlf{\fboxsep+\fboxrule}  
    % Add to it the distance to the box, and the width of the line of the box
\hspace{-\dlf}  
    % Move everything dlf units to the left, so that & #1 #2 is aligned under #1 & #2
\boxed{#1 #2}
    % Put a box around lhs and rhs
}
}


\newtheorem{thm}{Theorem}[section]
\newtheorem{lem}[thm]{Lemma}
\newtheorem{prop}[thm]{Proposition}
\newtheorem{cor}[thm]{Corollary}
\newtheorem{conj}[thm]{Conjecture}

\setcounter{secnumdepth}{4}

\titleformat{\paragraph}
{\normalfont\normalsize\bfseries}{\theparagraph}{1em}{}
\titlespacing*{\paragraph}
{0pt}{3.25ex plus 1ex minus .2ex}{1.5ex plus .2ex}

\definecolor{myblue}{RGB}{72, 165, 226}
\definecolor{myorange}{RGB}{222, 141, 8}

\setlength{\heavyrulewidth}{1.5pt}
\setlength{\abovetopsep}{4pt}


\DeclareMathOperator{\id}{id}
\DeclareMathOperator{\argmin}{\arg\!\min}
\DeclareMathOperator{\Tr}{Tr}

\newcommand{\bd}[1]{\mathbf{#1}} % for bolding symbols
\newcommand{\RR}{\mathbb{R}} % for Real numbers
\newcommand{\ZZ}{\mathbb{Z}} % for Integers
\newcommand{\col}[1]{\left[\begin{matrix} #1 \end{matrix} \right]}
\newcommand{\comb}[2]{\binom{#1^2 + #2^2}{#1+#2}}
\newcommand{\bs}{\boldsymbol}
\newcommand{\opn}{\operatorname}
\begin{document}
\nocite{*}
% \fontfamily{ptm}
% \renewcommand\rmdefault{ptm}

\title{Done \& To-Be-Done}

\author{Daeyoung Lim\thanks{Prof. Taeryon Choi} \\
Department of Statistics \\
Korea University}

\maketitle
\section{Done}
  \begin{enumerate}
    \item Normal
      \begin{itemize}
        \item Sparse GP + normal random effect + normal error
        \item Cosine basis + normal random effect + normal error
      \end{itemize}
    \item Binary
      \begin{itemize}
        \item Sparse GP + normal random effect: probit
        \item Cosine basis + normal random effect: probit
        \item Cosine basis + normal random effect: logistic
      \end{itemize}
  \end{enumerate}
  \section{To-Be-Done}
    \begin{enumerate}
      \item GLM
        \begin{enumerate}
          \item Normal
            \begin{itemize}
              \item Cosine basis + DPM random effect + normal error
            \end{itemize}
          \item Binary
            \begin{itemize}
              \item Cosine basis + DPM random effect: probit
              \item Cosine basis + DPM random effect: logistic
            \end{itemize}
          \item Poisson
            \begin{itemize}
              \item Cosine basis + normal random effect
              \item Cosine basis + DPM random effect
            \end{itemize}
          \item Negative Binomial
            \begin{itemize}
              \item Cosine basis + normal random effect
              \item Cosine basis + DPM random effect
            \end{itemize}
          \item All the above
            \begin{itemize}
              \item Apply shape restriction
            \end{itemize}
        \end{enumerate}
      \item Functional Data Analysis
        \begin{enumerate}
          \item Function-on-function
            \begin{itemize}
              \item Cosine basis + matrix normal random effect + matrix normal error
              \item Cosine basis + matrix normal random effect + matrix DPM error
              \item Cosine basis + matrix DPM random effect + normal error
              \item Cosine basis + matrix DPM random effect + matrix DPM error
            \end{itemize}
        \end{enumerate}
    \end{enumerate}
\end{document}