\documentclass[11pt]{article}

% This first part of the file is called the PREAMBLE. It includes
% customizations and command definitions. The preamble is everything
% between \documentclass and \begin{document}.

\usepackage[margin=0.6in]{geometry} % set the margins to 1in on all sides
\usepackage{graphicx} % to include figures
\usepackage{amsmath} % great math stuff
\usepackage{amsfonts} % for blackboard bold, etc
\usepackage{amsthm} % better theorem environments
% various theorems, numbered by section
\usepackage{amssymb}
\usepackage[utf8]{inputenc}
\usepackage{booktabs}
\usepackage{array}
\usepackage{courier}
\usepackage[usenames, dvipsnames]{color}
\usepackage{titlesec}
\usepackage{empheq}
\usepackage{tikz}
% \usepackage{pslatex}


\newcommand\encircle[1]{%
  \tikz[baseline=(X.base)] 
    \node (X) [draw, shape=circle, inner sep=0] {\strut #1};}
 
% Command "alignedbox{}{}" for a box within an align environment
% Source: http://www.latex-community.org/forum/viewtopic.php?f=46&t=8144
\newlength\dlf  % Define a new measure, dlf
\newcommand\alignedbox[2]{
% Argument #1 = before & if there were no box (lhs)
% Argument #2 = after & if there were no box (rhs)
&  % Alignment sign of the line
{
\settowidth\dlf{$\displaystyle #1$}  
    % The width of \dlf is the width of the lhs, with a displaystyle font
\addtolength\dlf{\fboxsep+\fboxrule}  
    % Add to it the distance to the box, and the width of the line of the box
\hspace{-\dlf}  
    % Move everything dlf units to the left, so that & #1 #2 is aligned under #1 & #2
\boxed{#1 #2}
    % Put a box around lhs and rhs
}
}


\newtheorem{thm}{Theorem}[section]
\newtheorem{lem}[thm]{Lemma}
\newtheorem{prop}[thm]{Proposition}
\newtheorem{cor}[thm]{Corollary}
\newtheorem{conj}[thm]{Conjecture}

\setcounter{secnumdepth}{4}

\titleformat{\paragraph}
{\normalfont\normalsize\bfseries}{\theparagraph}{1em}{}
\titlespacing*{\paragraph}
{0pt}{3.25ex plus 1ex minus .2ex}{1.5ex plus .2ex}

\definecolor{myblue}{RGB}{72, 165, 226}
\definecolor{myorange}{RGB}{222, 141, 8}

\setlength{\heavyrulewidth}{1.5pt}
\setlength{\abovetopsep}{4pt}


\DeclareMathOperator{\id}{id}
\DeclareMathOperator{\argmin}{\arg\!\min}
\DeclareMathOperator{\Tr}{Tr}

\newcommand{\bd}[1]{\mathbf{#1}} % for bolding symbols
\newcommand{\RR}{\mathbb{R}} % for Real numbers
\newcommand{\ZZ}{\mathbb{Z}} % for Integers
\newcommand{\col}[1]{\left[\begin{matrix} #1 \end{matrix} \right]}
\newcommand{\comb}[2]{\binom{#1^2 + #2^2}{#1+#2}}
\newcommand{\bs}{\boldsymbol}
\newcommand{\opn}{\operatorname}
\begin{document}
\nocite{*}
% \fontfamily{ptm}
% \renewcommand\rmdefault{ptm}

\title{Rosen's paper review}

\author{Daeyoung Lim\thanks{Prof. Taeryon Choi} \\
Department of Statistics \\
Korea University}

\maketitle
\section{Model Specifications}
  \begin{equation}
    \bs{y}_{i}\left(t_{ij}\right) = \bs{X}_{ij}\bs{\mu}\left(t_{ij}\right) + \bs{Z}_{ij}\bs{g}_{i}\left(t_{ij}\right) + \bs{\delta}_{i}\left(t_{ij}\right)
  \end{equation}
  where
  \begin{align}
    \bs{y}_{i}\left(t_{ij}\right)&:\; p\times 1 \qquad i=1,\ldots, n, \; j = 1,\ldots, m_{i}\\
    \bs{\mu}\left(t\right) &= \left(\bs{\mu}'_{1}\left(t\right), \ldots , \bs{\mu}'_{p}\left(t\right)\right)'\\
    \bs{\mu}'_{k}\left(t\right) &= \left(\mu_{k1}\left(t\right),\ldots , \mu_{kr}\left(t\right) \right)' : &r\times 1\\
    \bs{g}_{i}\left(t\right) &= \left(\bs{g}'_{i1}\left(t\right),\ldots , \bs{g}'_{ip}\left(t\right)\right)'\\
    \bs{g}_{ik}\left(t\right) &= \left(g_{ij1}\left(t\right),\ldots, g_{iks}\left(t\right)\right)':  &s \times 1\\
    \bs{x}_{ij}&:\; r \times 1\\
    \bs{z}_{ij}&:\; s\times 1\\
    \bs{X}_{ij} &= \bs{I}_{p} \otimes \bs{x}_{ij}'\\
    \bs{Z}_{ij} &= \bs{I}_{p} \otimes \bs{z}_{ij}'\\
    \bs{\delta}_{i}\left(t_{ij}\right) &\sim \text{Ornstein-Uhlenbeck process}
  \end{align}
\section{Ornstein-Uhlenbeck process}
An Ornstein-Uhlenbeck process is the second-order stationary process $\left\{X_{t}\right\}$, that satisfies the following differential equation:
\begin{equation}
  dX\left(t\right) = -aX\left(t\right)\,dt + \sigma\,dB\left(t\right), \quad t\geq 0
\end{equation}
where $\left\{B\left(t\right)\right\}$ is standard Brownian motion, and $a$ and $\sigma > 0$ are parameters and $X_{0}$ is a random variable that is independent of $\left\{B\left(t\right) \right\}$.
\subsection{Multivariate Ornstien-Uhlenbeck process}
The univariate version of the OU process naturally evolves to a multivariate form which reads as follows:
\begin{equation}
  d\bs{X}\left(t\right) = -\bs{A}\bs{X}\left(t\right)\,dt + \bs{B}\,d\bs{W}\left(t\right),
\end{equation}
where $\bs{A}, \bs{B}$ are constant matrices. The solution for this SDE is
\begin{equation}
  \bs{X}\left(t\right) = \exp\left(-\bs{A}t\right)\bs{X}_{0} + \int_{0}^{t}\exp\left\{-\bs{A}\left(t-t'\right) \right\}\bs{B}\,d\bs{W}\left(t\right).
\end{equation}
The properties of such an OU process are as folows:
\begin{itemize}
  \item If $\bs{A}$ has only eigenvalues with positive real part, a stationary solution exists of the form
  \begin{equation}
    \bs{X}_{s}\left(t\right) = \int_{-\infty}^{t}\exp\left\{-\bs{A}\left(t-t'\right) \right\}\bs{B}\,d\bs{W}\left(t\right).
  \end{equation}
  The expected value $\opn{E}\left[\bs{X}_{s}\left(t\right)\right] = 0$ and the covariance matrix
  \begin{equation}
    \bs{\Sigma} = \opn{Cov}\left(\bs{X}_{s}\left(t\right), \bs{X}_{s}'\left(s\right)\right) = \int_{-\infty}^{\min\left(t,s\right)}\exp\left\{-\bs{A}\left(t-t'\right) \right\}\bs{BB}'\exp\left\{-\bs{A}'\left(s-t'\right) \right\}\,dt'.
  \end{equation}
  \item The stationary covariance matrix satisfies the following equation:
  \begin{equation}
    \bs{A\Sigma} + \bs{\Sigma A}' = \bs{BB}'.
  \end{equation}
  The solution to this equation is given by
  \begin{equation}
    \bs{\Sigma} = \frac{\left|\bs{A}\right|\bs{BB}' + \left[\bs{A}-\Tr\left(\bs{A}\right)\bs{I} \right]\bs{BB}'\left[\bs{A}-\Tr\left(\bs{A}\right)\bs{I}\right]}{2\left(\Tr\left(\bs{A}\right)\right)\left|\bs{A}\right|}
  \end{equation}
  \item \emph{(Time Correlation Matrix in the Stationary State)} The following relations hold.
  \begin{align}
    \opn{Cov}\left(\bs{X}_{s}\left(t\right), \bs{X}'_{s}\left(s\right)\right) &= \exp\left\{-\bs{A}\left(t-s\right) \right\}\bs{\Sigma}, & t>s \\
    &= \bs{\Sigma}\exp\left\{-\bs{A}'\left(s-t\right) \right\}, & t<s
  \end{align}
  \item If all its transition densities depend only on the time differences, then the Markov process is \emph{homogeneous}. OU process is homogeneous. Therefore, the transition probabiliy of an OU process is given by
  \begin{equation}
    \opn{P}\left(\bs{X}_{s}\left(t\right)|\bs{X}_{s}\left(s\right), t-s\right) = \left|2\pi\bs{\Omega}\right|^{-1/2}\exp\left\{-\frac{1}{2}\bs{\gamma}'\bs{\Omega}^{-1}\bs{\gamma} \right\}
  \end{equation}
  where
  \begin{align}
    \bs{\gamma} &= \bs{X}_{s}\left(t\right) -\exp\left(-\bs{A}\left(t-s\right)\right)\bs{X}_{s}\left(s\right) \\
    \bs{\Omega} &=  \bs{\Sigma} -\exp\left\{-\bs{A}\left(t-s\right) \right\}\bs{\Sigma}\exp\left\{-\bs{A}'\left(t-s\right)\right\}.
  \end{align}
\end{itemize}
\section{Cubic Spline}
\end{document}