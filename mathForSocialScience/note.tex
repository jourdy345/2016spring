\documentclass[11pt]{article}

% This first part of the file is called the PREAMBLE. It includes
% customizations and command definitions. The preamble is everything
% between \documentclass and \begin{document}.

\usepackage[margin=0.6in]{geometry} % set the margins to 1in on all sides
\usepackage{graphicx} % to include figures
\usepackage{amsmath} % great math stuff
\usepackage{amsfonts} % for blackboard bold, etc
\usepackage{amsthm} % better theorem environments
% various theorems, numbered by section
\usepackage{amssymb}
\usepackage[utf8]{inputenc}
\usepackage{booktabs}
\usepackage{array}
\usepackage{courier}
\usepackage[usenames, dvipsnames]{color}
\usepackage{titlesec}
\usepackage{empheq}
\usepackage{tikz}

\newcommand\encircle[1]{%
  \tikz[baseline=(X.base)] 
    \node (X) [draw, shape=circle, inner sep=0] {\strut #1};}
 
% Command "alignedbox{}{}" for a box within an align environment
% Source: http://www.latex-community.org/forum/viewtopic.php?f=46&t=8144
\newlength\dlf  % Define a new measure, dlf
\newcommand\alignedbox[2]{
% Argument #1 = before & if there were no box (lhs)
% Argument #2 = after & if there were no box (rhs)
&  % Alignment sign of the line
{
\settowidth\dlf{$\displaystyle #1$}  
    % The width of \dlf is the width of the lhs, with a displaystyle font
\addtolength\dlf{\fboxsep+\fboxrule}  
    % Add to it the distance to the box, and the width of the line of the box
\hspace{-\dlf}  
    % Move everything dlf units to the left, so that & #1 #2 is aligned under #1 & #2
\boxed{#1 #2}
    % Put a box around lhs and rhs
}
}


\newtheorem{thm}{Theorem}[section]
\newtheorem{lem}[thm]{Lemma}
\newtheorem{prop}[thm]{Proposition}
\newtheorem{cor}[thm]{Corollary}
\newtheorem{conj}[thm]{Conjecture}

\setcounter{secnumdepth}{4}

\titleformat{\paragraph}
{\normalfont\normalsize\bfseries}{\theparagraph}{1em}{}
\titlespacing*{\paragraph}
{0pt}{3.25ex plus 1ex minus .2ex}{1.5ex plus .2ex}

\definecolor{myblue}{RGB}{72, 165, 226}
\definecolor{myorange}{RGB}{222, 141, 8}

\setlength{\heavyrulewidth}{1.5pt}
\setlength{\abovetopsep}{4pt}


\DeclareMathOperator{\id}{id}
\DeclareMathOperator{\argmin}{\arg\!\min}
\DeclareMathOperator{\Tr}{Tr}

\newcommand{\bd}[1]{\mathbf{#1}} % for bolding symbols
\newcommand{\RR}{\mathbb{R}} % for Real numbers
\newcommand{\ZZ}{\mathbb{Z}} % for Integers
\newcommand{\col}[1]{\left[\begin{matrix} #1 \end{matrix} \right]}
\newcommand{\comb}[2]{\binom{#1^2 + #2^2}{#1+#2}}
\newcommand{\bs}{\boldsymbol}
\newcommand{\opn}{\operatorname}
\begin{document}
\nocite{*}

\title{Math For Social Sciences}

\author{Daeyoung Lim\thanks{Prof. Junghwan Choi} \\
Department of Statistics \\
Korea University}

\maketitle

\section{Functions}
\subsection{Definition}
Given two sets $A, B$, a rule that maps $\forall x \in A$ to $y \in B$ is called a \emph{function}. The standard notation is
$$
  y = f\left(x\right), f:x\mapsto y\text{ or } f:A\mapsto B.
$$

\subsection{Inverse function}
For an inverse function to be defined, the original function should be one-to-one by which switching the domain and range still satisfies the definition of a function.

\begin{align*}
  f\left(x\right) &= \underbrace{\frac{f\left(x\right) +f\left(-x\right)}{2}}_{f_{1}\left(x\right)} +\underbrace{\frac{f\left(x\right)-f\left(-x\right)}{2}}_{f_{2}\left(x\right)}\\
  &= f_{1}\left(x\right) +f_{2}\left(x\right)\\
  f_{1}\left(-x\right) &= \frac{f\left(-x\right)+f\left(x\right)}{2} = f_{1}\left(x\right) \\
  f_{2}\left(-x\right) &= \frac{f\left(-x\right) -f\left(x\right)}{2} = -f_{2}\left(x\right)
\end{align*}

\subsection{Polynomials}
$$
  P\left(x\right) = a_{n}x^{n} + a_{n-1}a^{n-1} + \cdots +a_{2}x^{2} + a_{1}x + a_{0},\quad n \in \mathbb{Z}_{+}
$$
There are several properties of polynomials.
\begin{enumerate}
  \item $P\left(x\right) = x^{n}$ is an even function if $x = 2k$ and an odd function if $n = 2k +1$.
  \item $P\left(a\right) = 0 \implies \left(x-a\right)|P\left(x\right)$.
  \item \emph{(Weierstrauss approximation theorem)} If a function $f\left(x\right)$ is continuous on a closed interval $\left[a,b\right]$, then there exists an approximation for $f\left(x\right)$ on the same closed interval $\left[a,b\right]$.
\end{enumerate}
Suppose there are two circles with radii $r_{1}, r_{2}$ respectively such that $r_{1} > r_{2}$. If the same length $\Delta$ was added to the circumferences of the two circles, how larger have the radii of both circles become?
\begin{align*}
  \frac{2\pi r_{1} + \Delta}{2\pi} &= r_{1}'\\
  \frac{2\pi r_{2} + \Delta}{2\pi} &= r_{2}'\\
  r_{1}' - r_{1} &= \frac{\Delta}{2\pi}\\
  r_{2}' - r_{2} &= \frac{\Delta}{2\pi}
\end{align*}
Thus the lengths that are added to both radii are equal.

\begin{itemize}
  \item The formula for finding the roots of a polynomial exists up to the $4^{\text{th}}$ order. Starting from $5^{\text{th}}$ order, Galoi and Abel proved that there does not exist a fixed formula that expresses the roots with respect to the coefficients of a polynomial. 
  \item Suppose there are $(n+1)$ data points. How can we find a $n^{\text{th}}$ order polynomial that passes through all the data points? Formally in mathematical notation,
  $$
    P_{n}\left(x\right) = a_{n}x^{n} + a_{n-1}x^{n-1} + \cdots + a_{1}x + a_{0}
  $$
  and $\left\{\left(x_{1}, y_{1}\right), \left(x_{2}, y_{2}\right), \ldots , \left(x_{n}, y_{n}\right) \right\}$ is the set of all data points. Let's think of the following:
  \begin{align*}
    L_{n,0}\left(x\right) &= \frac{\left(x-x_{1}\right)\left(x-x_{2}\right)\cdots \left(x-x_{n}\right)}{\left(x_{0}-x_{1}\right)\left(x_{0}-x_{2}\right)\cdots \left(x_{0}-x_{n}\right)}\\
    L_{n,0}\left(x_{0}\right) &= 1\\
    L_{n,0}\left(x_{1}\right) &= 0\\
    L_{n,0}\left(x_{2}\right) &= 0\\
    \therefore L_{n,0}\left(x_{i}\right) &= \begin{cases} 1, & \text{if $i=0$}\\
    0, & \text{if $i\neq 0$} \end{cases}\\
    L_{n,1}\left(x_{i}\right) &= \begin{cases} 1, & \text{if $i=1$}\\
    0, & \text{if $i \neq i$} \end{cases}\\
    L_{n,j}\left(x_{i}\right) &= \begin{cases} 1, & \text{if $i=j$}\\
    0, & \text{if $i \neq j$} \end{cases}
  \end{align*}
  Now let's define the following:
  \begin{align*}
    L_{n}\left(x\right) &= y_{0}L_{n, 0}\left(x\right) + y_{1}L_{n,1}\left(x\right) + \cdots + y_{n}L_{n,n}\left(x\right)\\
    &= \sum_{i=0}^{n} L_{n,i}\left(x\right)y_{i}
  \end{align*}
This polynomial has been named after \emph{Lagrange}. 

$$
  P\left(x\right)|_{x=a}  = 0
$$
\end{itemize}
\end{document}